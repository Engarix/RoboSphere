\section{Motywacja i tło projektu}

Współczesna robotyka mobilna poszukuje rozwiązań pozwalających na poruszanie się w trudnych, 
nieustrukturyzowanych środowiskach. Klasyczne roboty kołowe, mimo swojej prostoty, 
często borykają się z problemem wywrócenia się lub zablokowania na przeszkodach. 

\vspace{0.5cm}

Robot kulisty, dzięki swojej specyficznej budowie, oferuje unikalną odporność na tego typu zdarzenia. 
Obudowa jest jednocześnie ochroną oraz elementem trakcyjnym, co sprawia, 
że jest on naturalnie chroniony przed czynnikami zewnętrznymi.

\vspace{0.5cm}

Aktualność tematyki robotów kulistych potwierdzają najnowsze wdrożenia w obszarze bezpieczeństwa publicznego. 
Przykładem pionierskiego rozwiązania jest chiński robot patrolowy RT-G, 
który na przełomie 2024 i 2025 roku rozpoczął służbę w prowincji Zhejiang. 
Urządzenie to, podobnie jak projekt opisany w niniejszej pracy, 
wykorzystuje mechanizm wewnętrznego wahadła do poruszania się, 
co pozwala mu na osiąganie dużych prędkości oraz sprawne manewrowanie zarówno na lądzie, 
jak i w środowisku wodnym. 
Zastosowanie napędu wahadłowego w robotach patrolowych pozwala na całkowitą izolację układów elektronicznych 
oraz eliminuje ryzyko przewrócenia się maszyny, co czyni go idealnym narzędziem do 
monitorowania trudnodostępnych przestrzeni miejskich.

\section{Opis problemu technicznego}

Głównym wyzwaniem w projektowaniu robota kulistego jest realizacja napędu przy zachowaniu szczelności i 
integralności sfery. Zastosowanie mechanizmu z wahadłem 
pozwala na zmianę położenia środka ciężkości wewnątrz kuli, co indukuje moment obrotowy i wymusza ruch toczny. 
Sterowanie takim układem jest nietrywialne pod kątem matematycznym, 
gdyż wymaga precyzyjnego operowania masą wahadła w celu uzyskania pożądanego wektora ruchu.

\section{Cel i zakres pracy}

Celem niniejszej pracy jest zaprojektowanie oraz budowa prototypu robota kulistego sterowanego za pomocą 
wewnętrznego układu wahadłowego, opartego na jednostce Raspberry Pi Zero 2W. Zakres pracy obejmuje:

\begin{itemize}
\item Opracowanie architektury sprzętowej wykorzystującej piny PWM do sterowania silnikami prądu stałego.
\item Zaprojektowanie mechanicznej konstrukcji robota
\item Implementację autorskiego protokołu komunikacyjnego opartego na gniazdach TCP/IP.
\item Stworzenie algorytmów kinematyki napędu różnicowego z uwzględnieniem płynnego narastania prędkości w celu ochrony mechanicznych elementów układu.
\item Analizę stabilności połączenia i bezpieczeństwa pracy robota poprzez mechanizmy kontroli stanu klienta.
\end{itemize}