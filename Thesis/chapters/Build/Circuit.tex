\section{Moduł zasilający}

\begin{figure}[ht]
    \centering
    \includegraphics[width=0.8\textwidth]{images/powerModule.png}
    \caption{Moduł zasilający}
    \label{fig:powerModule}
\end{figure}

\section{Komputer i silniki}

Fundamentem technicznym robota jest integracja wydajnej jednostki obliczeniowej z precyzyjnym układem wykonawczym.

\subsection{Jednostka centralna – Raspberry Pi Zero 2W}

Sercem układu sterowania jest minikomputer Raspberry Pi Zero 2W. Wybór tej platformy wynika z jej unikalnych cech:
\begin{itemize}
    \item \textbf{Kompaktowe wymiary:} Niewielka obudowa pozwala na montaż elektroniki bezpośrednio na ramie wahadła, co jest kluczowe dla zachowania wyważenia wewnętrznego mechanizmu.
    \item \textbf{Łączność bezprzewodowa:} Zintegrowany moduł Wi-Fi umożliwia komunikację z komputerem operatora bez potrzeby stosowania zewnętrznych adapterów.
    \item \textbf{Możliwości sprzętowe:} Wielordzeniowy procesor pozwala na jednoczesną obsługę stosu sieciowego TCP/IP oraz generowanie precyzyjnych sygnałów sterujących silnikami.
\end{itemize}

\subsection{Układ wykonawczy i mostki H L293D}

Do sterowania silnikami prądu stałego wykorzystano zintegrowane mostki H typu \textbf{L293D}. Wybór ten umożliwia niezależne sterowanie kierunkiem oraz prędkością obrotową dwóch silników przez jeden układ scalony.



Mostek L293D pracuje w logice cztero-kanałowej i umożliwia:
\begin{itemize}
    \item \textbf{Kontrolę kierunku:} Poprzez podanie stanów wysokich i niskich na wejścia binarne (piny \textit{Input}), co odpowiada konfiguracji pinów \textit{Forward} i \textit{Backward} w projekcie.
    \item \textbf{Regulację prędkości:} Wykorzystanie pinu \textit{Enable} do podania sygnału PWM (Pulse Width Modulation), co pozwala na płynne zarządzanie mocą przekazywaną na silniki.
    \item \textbf{Zabezpieczenie układu:} Wbudowane diody zabezpieczające chronią elektronikę sterującą przed przepięciami indukowanymi przez cewki silników podczas ich zatrzymywania lub zmiany kierunku.
\end{itemize}

\subsection{Napęd i zasilanie}

Napęd robota opiera się na trzech silnikach DC: dwóch trakcyjnych, realizujących ruch toczny kuli, oraz jednym silniku wahadła. Ten ostatni, poprzez zmianę położenia masy wewnętrznej, pozwala na dynamiczne sterowanie środkiem ciężkości, co jest kluczowe dla manewrowania jednostką sferyczną.

\begin{figure}[ht]
    \centering
    \includegraphics[width=0.8\textwidth]{images/hardware.png}
    \caption{Komputer i silniki}
    \label{fig:hardware}
\end{figure}

