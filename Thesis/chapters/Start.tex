\section{Inicjalizacja i procedura startowa systemu}

Prawidłowe funkcjonowanie robota sferycznego wymaga skoordynowanej inicjalizacji wszystkich warstw systemowych – od niskopoziomowej konfiguracji pinów GPIO, przez logikę sterowania, aż po warstwę komunikacji sieciowej. Proces ten został zaprojektowany tak, aby zapewnić maksymalne bezpieczeństwo sprzętu już od momentu podania zasilania.

\subsection{Inicjalizacja warstwy sprzętowej (Hardware Layer)}

Pierwszym etapem po uruchomieniu skryptu głównego na jednostce Raspberry Pi jest inicjalizacja obiektu klasy \texttt{RobotDriver}. W tym kroku system wykonuje następujące operacje:
\begin{itemize}
    \item \textbf{Konfiguracja pinów:} Na podstawie pliku \texttt{robotSettings.py} przypisywane są funkcje dla poszczególnych linii GPIO. Wykorzystanie biblioteki \texttt{gpiozero} \cite{gpiozero_docs} pozwala na automatyczne ustawienie pinów w bezpiecznych stanach niskich, co zapobiega niekontrolowanym szarpnięciom silników przy starcie.
    \item \textbf{Instancjonowanie napędów:} Tworzone są trzy obiekty klasy \texttt{Motor} (lewy, prawy oraz wahadło). Dla każdego z nich inicjalizowane są dwa wyjścia cyfrowe (kierunek) oraz jedno wyjście z modulacją PWM (prędkość).
    \item \textbf{Reset kinematyki:} Obiekt \texttt{DiffDriveKinematics} zeruje wszystkie wewnętrzne bufory prędkości (\texttt{curr\_left}, \texttt{curr\_right}, \texttt{curr\_pend}), przygotowując algorytm rampingu do płynnego startu od wartości zero.
\end{itemize}

\subsection{Uruchomienie jednostki sterującej i serwera}

Po pomyślnym przygotowaniu sterowników, system przechodzi do inicjalizacji logiki wysokopoziomowej w module \texttt{Controller}. Kontroler ten staje się pośrednikiem między siecią a sprzętem.

Ostatnim ogniwem procedury startowej jest uruchomienie \texttt{CommandServer}. Serwer otwiera gniazdo TCP/IP na porcie 8080 i przechodzi w stan nasłuchiwania (\textit{listening}). W tym momencie robot jest widoczny w sieci lokalnej i oczekuje na pakiet inicjalizujący od aplikacji klienckiej (komputer operatora).



\subsection{Ustanowienie połączenia i autoryzacja}

Pełna operacyjność systemu następuje po wykonaniu tzw. "uścisku dłoni" (ang. \textit{handshake}) z kontrolerem zewnętrznym:
\begin{enumerate}
    \item Klient nawiązuje połączenie TCP.
    \item Serwer inicjalizuje obiekt \texttt{ClientState}, ustawiając znacznik czasu \texttt{last\_seen} na wartość aktualną.
    \item Pierwsza odebrana komenda \texttt{PING} potwierdza drożność kanału komunikacyjnego, na co serwer odpowiada komunikatem \texttt{PONG}, sygnalizując gotowość do przyjmowania komend \texttt{MOVE}.
\end{enumerate}

Wprowadzenie tej wieloetapowej inicjalizacji pozwala na uniknięcie błędów wynikających z prób sterowania niegotowym sprzętem oraz gwarantuje, że robot zareaguje na polecenia tylko wtedy, gdy wszystkie podsystemy pracują poprawnie.